\documentclass[11pt]{article}
\usepackage[utf8]{inputenc} 
\usepackage[T1]{fontenc} % fonts to encode unicode
\usepackage{times}
\usepackage{url}
\sloppy
\hyphenpenalty 10000
\usepackage{xcolor}
% for Letter size

%\setlength\topmargin{0.2cm} \setlength\oddsidemargin{-0cm}
%\setlength\textheight{22cm} \setlength\textwidth{15.8cm}
%\setlength\columnsep{0.25in}  \newlength\titlebox \setlength\titlebox{2.00in}
%\setlength\headheight{5pt}   \setlength\headsep{0pt}
%\setlength\footskip{1.0cm}
%\setlength\leftmargin{0.0in}
%\pagestyle{empty}
%%%%%%%%%%%%%%%%%%%%%%%%%%%%%%%%%%%%%%%%%%%%%%%%%%%%%%%%%%%%%%%%%%%%%%%%%%%


% for A4 size

\setlength\topmargin{-5mm} \setlength\oddsidemargin{-0cm}
\setlength\textheight{24.7cm} \setlength\textwidth{16cm}
\setlength\columnsep{0.6cm}  \newlength\titlebox \setlength\titlebox{2.00in}
\setlength\headheight{5pt}   \setlength\headsep{0pt}
\setlength\footskip{1.0cm}
\setlength\leftmargin{0.0in}
\pagestyle{empty}
%%%%%%%%%%%%%%%%%%%%%%%%%%%%%%%%%%%%%%%%%%%%%%%%%%%%%%%%%%%%%%%%


\setlength{\parindent}{0in}
\setlength{\parskip}{2ex}

\newcommand\witiii{WIT\raisebox{.5ex}{\small3}}

\begin{document}

\begin{center}
\Large{\textbf{Proposal for the Fourth Workshop on Discourse in Machine Translation DiscoMT 2019}}
\end{center}

\section{Description and justification}

\textcolor{magenta}{2 pages max for description only}

\textcolor{magenta}{- Goal to wide context in MT.} 

\textcolor{magenta}{- Lots of work on context-wide MT, but community is lacking cohesion somehow.}

\textcolor{magenta}{- Goal to promote that people work on the same body of resources: those on the repository https://bitbucket.org/hy-crossNLP/discomt/wiki/Home } 

\textcolor{magenta}{- Weight on more linguistic stuff}

\textcolor{magenta}{- Throw out rule-based, start history with SMT, add paragraph what's going on on NMT}

\textcolor{magenta}{- Work on MT evaluation through test suites that has been going on}

\textcolor{magenta}{- Human parity discussion that has been going on }

\textcolor{magenta}{- Idea of invited speakers: invite them, funding... Potential names: Kellie Webster (Google), Andy Way (Dublin)}

\textcolor{magenta}{Website should be up an running. }

When translating entire texts, as opposed to individual sentences only, 
one cannot ignore text-level properties. This is becoming increasingly
clear in neural machine translation (NMT), where text-level aspects of
translation may now be the main obstacle to high-quality automatic
translation for high-resource languages after the advances in translation
quality observed in recent years.\footnote{Samuel Läubli, Rico Sennrich
and Martin Volk (2018). Has machine translation achieved human parity? A case
for document-level evaluation. Proceedings of the 2018 Conference on
Empirical Methods in Natural Language Processing.} Approaches to this
problem include a plethora of recent work on NMT with extended context, but also
research on linguistically defined properties of MT such as

\begin{itemize}
\item document-wide properties, such as style, register,
  reading level and genre, all of which are manifest in the frequency
  and distribution of words, word senses, referential forms and
  syntactic structures; 
\item patterns of topical or functional sub-structure that show up
   in local differences in frequency and distribution;
\item patterns of discourse coherence, manifest in explicit and
  implicit relations between sentences (clauses), or between
  sentences (clauses) and referring forms, or between referring forms
  themselves (which include noun phrases and pronouns);
\item common use of short context-sensitive expressions that rely on
  context to convey a lot of information in very few words.
\end{itemize}

In the 1990s, these properties had stimulated considerable research in Machine
Translation, aimed at endowing machine-translated texts with similar document
and discourse properties as their source texts.
% This included work on stylistics for MT, target language realization of
% source-language discourse relations and of referring forms, anaphor resolution
% for generating appropriate target-language pronouns, and ellipsis resolution
% for generating appropriate target-language forms from ellipsed verb-phrases.
% Pointers to much of this work can be found in the \textit{Machine Translation
% Archive} of conference and workshop papers from the 1990s
% (\url{http://www.mt-archive.info/srch/ling-90.htm}).
% 
% This early period essentially ended with the 1999 publication of a special
% issue of the journal \textit{Machine Translation}, edited by Ruslan Mitkov,
% devoted to anaphora resolution in Machine Translation and multi-lingual NLP. A
% period of ten years then elapsed before interest resumed in these topics, now
% from the perspectives of Statistical and/or Hybrid Machine Translation.  
% 
In the literature on data-driven MT approaches there has been considerably more
work on discourse and machine translation since 2010, by an increasing number of
people and groups. The first three editions of DiscoMT have helped to
consolidate a small, but thriving community of researchers interested in
discourse-level MT. Moreover, the recent advances in NMT research have spawned
an emerging body of work on the exploitation of cross-sentence context from a
more machine learning-oriented perspective, with encouraging, if preliminary
results on some of the problems studied by the DiscoMT community.\footnote{Elena
Voita, Pavel Serdyukov, Rico Sennrich and Ivan Titov (2018). Context-Aware
Neural Machine Translation Learns Anaphora Resolutions. Proceedings of the 56th
Annual Meeting of the ACL.} A problem in this line of research is that the
approaches proposed so far have been evaluated on very different language pairs,
corpora and data sets, rendering it almost impossible to make a fair comparison
of the results in the literature. To address this problem, we propose to run an
``unshared task'' at DiscoMT 2019 to promote the evaluation of different
context-enabled MT system in comparable conditions.

Considerable effort has been expended recently on document-level MT, such that now, several
individuals and/or groups are working on similar or overlapping problems.
Notable efforts include work on document-level influences on lexical choice in SMT, methods of document-level
SMT, annotated resources for discourse-level SMT, discourse factors affecting translation quality,
discourse-sensitive assessment metrics, and specific discourse phenomena in SMT.

As exemplified by the papers presented at DiscoMT 2017, active research topics in document-level MT are: 
NMT extension taking into consideration context from multiple sentences or
entire documents;
pronoun translation between languages which differ in pronoun usage; 
explicitation/implicitation in translating discourse connectives; 
context-aware translation of ambiguous terms; assessing document-level properties of MT
output, including coherence; preserving document-level
properties characteristic of register, genre, and other types of text variation; and difficulties 
in preserving them in a MT frameworks relying on neural attention or word
alignment.

A critical mass of researchers for the DiscoMT 2015 and the WMT 2016 shared tasks
on pronoun-focused translation has been reached.  The advent of neural machine 
translation in the past four years has opened new opportunities for modeling 
text-level properties with deep neural networks, and we expect to see a number
of results presented and discussed at DiscoMT 2019.


\section{Previous editions of DiscoMT: 2013 and 2015}

\textcolor{magenta}{Move to the end. Include stats about how many participants we had before.} 

\textcolor{magenta}{Include DiscoMT2017.} 

The growing interest in discourse and MT led to the first
\textit{ACL Workshop on Discourse in Machine Translation (DiscoMT)} in 2013,
held in Sofia, Bulgaria.\footnote{Bonnie Webber  and  Andrei Popescu-Belis  and  Katja Markert  and  J\"{o}rg Tiedemann. (2013). \textit{Proceedings of the Workshop on Discourse in Machine Translation (DiscoMT)}. Sofia, Bulgaria. Association for Computational Linguistics. \url{http://www.aclweb.org/anthology/W13-33}.}
The workshop
featured eight papers on such topics as lexical consistency in human and
machine translation, lexical tightness as a whole-document
measure of machine-translated targets, improving
the translation of tensed verbs by recognizing whether or not they are
conveying narrative, and document-level decoding.  For the first edition,
there were 12 submissions, 6 of which were accepted as oral presentations and
2 as posters only.  Oral presenters were also encouraged to make posters
describing their work, and all resulting posters were presented in joint
poster session with WMT 2013, thus offering DiscoMT an increased visibility.

To allow for sufficient novelty and volume of research contributions, the
organizers decided to hold DiscoMT every two years.  For the second
\textit{EMNLP Workshop on Discourse in Machine Translation (DiscoMT 2015)},
held in Lisbon, Portugal,\footnote{Webber, B., Carpuat, M., Popescu-Belis, A., \& Hardmeier, C. (2015). \textit{Proceedings of the Second Workshop on Discourse in Machine Translation (DiscoMT 2015)}. Lisbon, Portugal. Association for Computational Linguistics.  \url{https://aclweb.org/anthology/W/W15/\#2500}.}, the number of submitted and accepted paper had increased.  We maintained the idea of a joint poster session with WMT.  Out of 15 submissions,
5 were accepted as long papers, 3 as short papers, and 4 as posters only.  Moreover, nine papers described shared task submissions and an overview of the two sub-tasks of the shared task.

\section{Shared tasks: DiscoMT 2015, WMT 2016, and DiscoMT 2017}

Pronoun translation is a problem that has been independently approached
by different researchers in the last few years, which led to the organization of two shared tasks, in 2015 and 2016.
Incorrect pronoun translations can have a large impact on the readability
of a text, and the overuse of masculine default pronouns in MT output is not
only a linguistic, but also a gender-political issue.
Pronoun use is governed by strong grammatical rules, so it is relatively
straightforward to evaluate system performance objectively. Finally, the
importance of the problem is easy to demonstrate, even to an audience that is
ignorant of discourse.

The DiscoMT 2015 shared task\footnote{Christian Hardmeier, Preslav Nakov, Sara Stymne, Jörg Tiedemann, Yannick Versley, and Mauro Cettolo. (2015). Pronoun-Focused MT and Cross-Lingual Pronoun Prediction: Findings of the 2015 DiscoMT Shared Task on Pronoun Translation.  \textit{Proceedings of the Second Workshop on Discourse in Machine Translation (DiscoMT)}, pages 1–16, Lisbon, Portugal, 17 September 2015.  Association for Computational Linguistics.}
included two subtasks, relevant to both the machine translation and the discourse communities: (i)~pronoun-focused translation, a practical
MT task, and (ii)~cross-lingual pronoun prediction, a classification task that required no specific MT expertise and is interesting as a machine learning task in its own right. The task focused on EN/FR, for which MT output is generally of high quality, but has visible issues with pronoun translation due
to differences in the pronoun systems of the two languages. Six groups participated in the pronoun-focused translation task and eight groups in the cross-lingual pronoun prediction task.

The WMT 2016 shared task on cross-lingual pronoun prediction\footnote{Guillou, Liane; Hardmeier, Christian; Nakov, Preslav; Stymne, Sara; Tiedemann, Jörg; Versley, Yannick; Cettolo, Mauro; Webber, Bonnie; Popescu-Belis, Andrei. (2016).  Findings of the 2016 WMT shared task on cross-lingual pronoun prediction, \textit{Proceedings of the First Conference on Machine Translation (WMT16)}, pages 522-539.  Berlin, Germany. Association for Computational Linguistics. \url{https://aclweb.org/anthology/W/W16/W16-2345.pdf}}
was a  classification  task  in which  participants  were  asked  to  provide predictions  on  what  pronoun  class  label should replace a placeholder value in the target-language  text,  provided  in  lemmatised and PoS-tagged form.  Four subtasks were offered, for EN/FR and EN/DE language  pairs,  in  both directions.   Eleven  teams  participated  in the  shared  task. Most of the submissions outperformed two strong language-model-based baseline systems, with systems using deep recurrent neural networks outperforming those using other architectures for most language pairs.

We foresee the organization of a shared task on pronoun translation at DiscoMT 2017.  Minutes have been taken after the concluding discussion at the WMT 2016 shared task, and plans have been made for a 2017 shared task.  The idea is to maintain a sub-task on EN/FR/DE pronoun prediction (given the source text and a lemmatized (or ``impoverished'') version of the target text stripped of its pronouns), but to consider again the possibility of a pronoun-focused translation task, while attempting to solve the evaluation issues (trade-off between cost and accuracy) related to it.

\section{Timeline}

\textcolor{magenta}{ACL: call for papers before Christmas and deadline 26 April, notification 24 May. Reviewing will happen in May then.} 

\section{Organizational Information}

The proposed workshop has a small set of \textbf{Co-Chairs}, with responsibility
for scientific and administrative matters involved in delivering a workshop.  The members of 
the \textbf{Programme Committee} will mainly be responsible for reviewing submissions, but 
they will also be consulted for matters related to the organization and
intellectual content of the workshop.  A separate group will have responsibility for the Shared Task,
and liaise with the Co-Chairs regarding in particular the quality of the system papers.  The responsible
person for the shared task will be a co-editor of the proceedings volume as well.

\medskip

\textbf{Co-Chairs:}
\begin{itemize}

\item Christian Hardmeier



\item Sharid Loáiciga

\item  Andrei Popescu-Belis <andrei.popescu-belis@idiap.ch>\\
Idiap Research Institute, Centre du Parc, Rue Marconi 19, PO Box 592,
1920 Martigny, Switzerland\\
\url{http://www.idiap.ch/~apbelis/}\\
Phone: +41 27 721 7729\\
Expertise/Interests: Andrei's research focuses on language processing at the discourse-level, 
in particular reference and anaphora resolution, discourse markers and
connectives, and dialogue acts; the evaluation of NLP and interactive
systems, including machine translation systems and meeting browsers;
and language and multimodal resources.


\end{itemize}

\medskip

\textbf{Tentative Programme Committee}

\textcolor{magenta}{75\% should be approved}

\begin{itemize}
\item Beata Beigman Klebanov, Educational Testing Service, New Jersey USA
\item Thomas Brovelli, Google, Zurich, Switzerland
\item Marine Carpuat, University of Washington, Seattle, US
\item Mauro Cettolo, Fondazione Bruno Kessler, Trento, Italy
\item Marta Costa-jussà, Universitat Politècnica de Catalunya, Spain (confirmed)
\item Zhengxian Gong, Soochow University, China (confirmed)
\item Liane Guillou, University of Edinburgh, UK
\item Francisco Guzm\'{a}n, Qatar Computing Research Institute, Doha, Qatar
\item Gholamreza Haffari, Monash University, Clayton, Australia
\item Shafiq Joty, Qatar Computing Research Institute, Doha, Qatar (confirmed)
\item Ekaterina Lapshinova-Kultunsky, Saarland University, Saarbrücken, Germany 
\item Lori Levin, Language Technologies Institute, Carnegie Mellon University, Pittsburgh, PA, USA
\item Lesly Miculicich-Werlen, Idiap Research Institute, Martigny, Switzerland (confirmed)
\item Preslav Nakov, Qatar Computing Research Institute, Doha, Qatar (confirmed)
\item Ani Nenkova, University of Pennsylvania, USA
\item Michal Novak, Charles University, Prague CZ
\item Lucie Pol\'{a}kov\'{a}, Charles University, Prague CZ
\item Maja Popovic, DFKI, Berlin DE
\item Annette Rios, University of Zurich, Switzerland
\item Rico Sennrich, University of Edinburgh, UK (confirmed)
\item Lucia Specia, University of Sheffield, UK
\item Sara Stymne, Uppsala University, Uppsala, Sweden
\item Jörg Tiedemann, University of Helsinki, Finland (confirmed)
\item Yannick Versley, University of Heidelberg, Heidelberg, Germany
\item Martin Volk, University of Zurich, Zurich, Switzerland
\item Bonnie Webber, University of Edinburgh, UK
\item Kellie Webster, Google, New York, US (confirmed)
\item Min Zhang, Soochow University, Suzhou, China
\item Sandrine Zufferey, University of Bern, Switzerland
\end{itemize}

\section{Invited Speakers and sources for funding}



\section{Diversity}

Account of efforts for diversity of speakers, program committee members and participants

Balance invited speakers

Balance committee: gender, geography, community is very Europe-oriented, but active efforts to recruit people from outside Europe, compared to previous years. 



\section{Other Information}

\begin{itemize}
\item \textbf{Proposed workshop length:} One day (Ideally, this would coincide with
the second day of the Workshop on Machine Translation, so that the two workshops could
share a poster session.)

\item \textbf{Estimated number of attendees:} 25-30 people

\item \textbf{Special requirements or technical needs:} None

\item \textbf{Preferred venue:} \textcolor{magenta}{ACL Florence, community is mostly in Europe but maily WMT will try to be there and we want to be together.} 
\end{itemize}


\end{document}
