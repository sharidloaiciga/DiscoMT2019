\documentclass[11pt]{article}
\usepackage[utf8]{inputenc} 
\usepackage[T1]{fontenc} % fonts to encode unicode
\usepackage{times}
\usepackage{url}
\sloppy
\hyphenpenalty 10000
\usepackage{xcolor}
% for Letter size

%\setlength\topmargin{0.2cm} \setlength\oddsidemargin{-0cm}
%\setlength\textheight{22cm} \setlength\textwidth{15.8cm}
%\setlength\columnsep{0.25in}  \newlength\titlebox \setlength\titlebox{2.00in}
%\setlength\headheight{5pt}   \setlength\headsep{0pt}
%\setlength\footskip{1.0cm}
%\setlength\leftmargin{0.0in}
%\pagestyle{empty}
%%%%%%%%%%%%%%%%%%%%%%%%%%%%%%%%%%%%%%%%%%%%%%%%%%%%%%%%%%%%%%%%%%%%%%%%%%%


% for A4 size

\setlength\topmargin{-5mm} \setlength\oddsidemargin{-0cm}
\setlength\textheight{24.7cm} \setlength\textwidth{16cm}
\setlength\columnsep{0.6cm}  \newlength\titlebox \setlength\titlebox{2.00in}
\setlength\headheight{5pt}   \setlength\headsep{0pt}
\setlength\footskip{1.0cm}
\setlength\leftmargin{0.0in}
\pagestyle{empty}
%%%%%%%%%%%%%%%%%%%%%%%%%%%%%%%%%%%%%%%%%%%%%%%%%%%%%%%%%%%%%%%%


\setlength{\parindent}{0in}
\setlength{\parskip}{2ex}

\newcommand\witiii{WIT\raisebox{.5ex}{\small3}}

\begin{document}

\begin{center}
\Large{\textbf{Proposal for the Fourth Workshop on Discourse in Machine Translation: DiscoMT 2019}}\\
\url{https://www.idiap.ch/workshop/DiscoMT}
\end{center}

% APB: as we need it for advertisement anyway, why shouldn't we put here the tagline?
% => A very brief advertisement or tagline for the workshop, up to 140 characters, that highlights any key information you wish prospective attendees to know, and which would be suitable to be put onto a web-based survey (see below).

% APB: following "plan" suggested in the call for workshops
% (not a very well structured "plan" !)

\section{Workshop Topic and Content}

%\textcolor{magenta}{2 pages max for description only}
%\textcolor{magenta}{- Goal to wide context in MT.} 
%\textcolor{magenta}{- Lots of work on context-wide MT, but community is lacking cohesion somehow.}
%\textcolor{magenta}{- Goal to promote that people work on the same body of resources: those on the repository https://bitbucket.org/hy-crossNLP/discomt/wiki/Home } 
%\textcolor{magenta}{- Weight on more linguistic stuff}
%\textcolor{magenta}{- Throw out rule-based, start history with SMT, add paragraph what's going on on NMT}
%\textcolor{magenta}{- Work on MT evaluation through test suites that has been going on}
%\textcolor{magenta}{- Human parity discussion that has been going on }


%\textcolor{magenta}{Website should be up an running. }

When translating entire texts, as opposed to individual sentences only, 
one cannot ignore text-level properties. This is becoming increasingly
clear in neural machine translation (NMT), where text-level aspects of
translation may now be the main obstacle to high-quality automatic
translation for high-resource languages after the advances in translation
quality observed in recent years.\footnote{Samuel Läubli, Rico Sennrich
and Martin Volk. Has machine translation achieved human parity? A case
for document-level evaluation. Proceedings of the 2018 Conference on
Empirical Methods in Natural Language Processing.} Approaches to this
probl

\begin{itemize}
\item document-wide properties, such as style, register,
  reading level and genre, all of which are manifest in the frequency
  and distribution of words, word senses, referential forms and
  syntactic structures; 
\item patterns of topical or functional sub-structure that show up
   in local differences in frequency and distribution;
\item patterns of discourse coherence, manifest in explicit and
  implicit relations between sentences (clauses), or between
  sentences (clauses) and referring forms, or between referring forms
  themselves (which include noun phrases and pronouns);
\item common use of short context-sensitive expressions that rely on
  context to convey a lot of information in very few words.
\end{itemize}

In the 1990s, these properties had stimulated considerable research in
Machine Translation, aimed at endowing machine-translated texts with
similar document and discourse properties as their source texts.
This included work on stylistics for MT, target language realization of 
source-language discourse
relations and of referring forms, anaphor resolution for generating
appropriate target-language pronouns, and ellipsis
resolution for generating appropriate target-language forms from
ellipsed verb-phrases. Pointers to much of this work can
be found in the \textit{Machine Translation Archive} of conference
and workshop papers from the 1990s (\url{http://www.mt-archive.info/srch/ling-90.htm}).

This early period essentially ended with the 1999 publication of a special
issue of the journal \textit{Machine Translation}, edited by Ruslan Mitkov,
devoted to anaphora resolution in Machine Translation and multi-lingual NLP.
A period of ten years then elapsed before interest resumed in these topics,
now from the perspectives of Statistical and/or Hybrid Machine
Translation.  

Since 2010, there has
been considerably more work on discourse and machine translation, by 
an increasing number of people and groups. Moreover, SMT
has itself evolved in ways that allow more access to needed linguistic knowledge,
through the feature-rich statistical models that have recently become available.
The branch of \textit{document-level machine translation} is
now well established, thanks in particular to the first two editions of
the DiscoMT (in 2013 and 2015). 

Considerable effort has been expended recently on document-level MT, such that now, several
individuals and/or groups are working on similar or overlapping problems.
Notable efforts include work on document-level influences on lexical choice in SMT, methods of document-level
SMT, annotated resources for discourse-level SMT, discourse factors affecting translation quality,
discourse-sensitive assessment metrics, and specific discourse phenomena in SMT.

As exemplified by the papers presented at DiscoMT 2015, active research topics in document-level MT are: 
pronoun translation between languages which differ in pronoun usage; 
explicitation/implicitation in translating discourse connectives; 
context-aware translation of ambiguous terms; assessing document-level properties of MT
output, including coherence; preserving document-level
properties characteristic of register, genre, and other types of text variation; and difficulties 
in preserving them in a purely alignment-based MT framework.

A critical mass of researchers for the DiscoMT 2015 and the WMT 2016 shared tasks
on pronoun-focused translation has been reached.  The advent of neural machine 
translation in the past two years has opened new opportunities for modeling 
text-level properties with deep neural networks, and we expect to see a number
of results presented and discussed at DiscoMT 2017.

%%\textcolor{magenta}{- Idea of invited speakers: invite them, funding... Potential names: Kellie Webster (Google), Andy Way (Dublin)}
% APB: the workshop will consist of invited talks and regular talks
% + explain why no there is shared task this time
% explain the idea of the joint poster session with WMT, and that all DiscoMT papers are also encouraged to do posters

% APB: explain here:
% To allow for sufficient novelty and volume of research contributions, the organizers decided to hold DiscoMT every two years. 

% \section{Timeline}
% \textcolor{magenta}{ACL: call for papers before Christmas and deadline 26 April, notification 24 May. Reviewing will happen in May then.} 
% APB: For the timeline, simply say that we will respect the one given by ACL!

%--------------------------------------------------------------


\section{Organizers}

%\textbf{How DiscoMT will be organized.}
DiscoMT 2019 will have three organizers, who are co-chairing the workshop.  They will have responsibility for scientific and administrative matters involved in delivering a high-quality workshop.  The members of the Program Committee listed below will mainly be responsible for reviewing submissions, but they will also be consulted for matters related to the organization and intellectual content of the workshop, and will help advertise the workshop.

% The names, affiliations, and email addresses of the organizers, with one-paragraph statements of their research interests, areas of expertise, and experience in organizing workshops and related events.

\textbf{Christian Hardmeier}


\textbf{Sharid Loáiciga}

\textbf{Andrei Popescu-Belis} (\texttt{andrei.popescu-belis@heig-vd.ch}) is a professor at the School of Management and Engineering Vaud (HEIG-VD), an institution that is part of the University of Applied Sciences and Arts of Western Switzerland (HES-SO).  He has been a group leader at the Idiap Research Institute, a senior researcher at the University of Geneva, and is a lecturer at the Swiss Federal Institute of Technology (EPFL).  His research focuses on language processing at the discourse-level, in particular reference and anaphora resolution, discourse markers and connectives, and dialogue acts.  Among his other interests are: the evaluation of NLP and interactive systems, including machine translation systems and meeting browsers; and language and multimodal resources.  He has been a co-chair of all past editions of DiscoMT, and serves annually as a reviewer for the major *ACL events.  He has also chaired two Machine Learning and Multimodal Interaction workshops and has been area chair for discourse at ACL 2012.


\section{Program Committee}

%\textcolor{magenta}{75\% should be approved}

\begin{itemize} \setlength{\itemsep}{0pt}
\item Beata Beigman Klebanov, Educational Testing Service, New Jersey, USA
%\item Thomas Brovelli, Google, Zurich, Switzerland %is he still doing research in MT?
\item Marine Carpuat, University of Washington, Seattle, USA
%\item Mauro Cettolo, Fondazione Bruno Kessler, Trento, Italy SL: he declined
\item Marta Costa-jussà, Universitat Politècnica de Catalunya, Spain (confirmed)
\item Zhengxian Gong, Soochow University, China (confirmed)
\item Liane Guillou, University of Edinburgh, UK
\item Francisco Guzm\'{a}n, Facebook, USA
\item Yulia Grishina, University of Postdam \& Amazon, Berlin, Germany (confirmed)
\item Gholamreza Haffari, Monash University, Clayton, Australia
\item Shafiq Joty, Nanyang Technological University, Singapore (confirmed)
\item Ekaterina Lapshinova-Kultunsky, Saarland University, Saarbrücken, Germany (confirmed)
%\item Lori Levin, Language Technologies Institute, Carnegie Mellon University, Pittsburgh, PA, USA % APB: is she still around and active? SL: not sure
\item Lesly Miculicich Werlen, Idiap Research Institute, Martigny, Switzerland (confirmed)
\item Preslav Nakov, Qatar Computing Research Institute, Doha, Qatar (confirmed)
\item Ani Nenkova, University of Pennsylvania, USA
\item Michal Novak, Charles University, Prague CZ (confirmed) % APB: are we still in contact with him? SL: yes, I saw him at Naacl actually, still working on coref
\item Nikolaos Pappas, Idiap Research Institute, Martigny, Switzerland (confirmed)
\item Lucie Pol\'{a}kov\'{a}, Charles University, Prague CZ (confirmed)% APB: are we still in contact with her? SL: I don't know her personally, but she accepted
%\item Maja Popovic, DFKI, Berlin DE SL: email bounced
\item Annette Rios, University of Zurich, Switzerland
\item Carol Scarton, University of Sheffield, UK (confirmed)
\item Rico Sennrich, University of Edinburgh, UK (confirmed)
%\item Lucia Specia, Imperial College London, UK SL: email bounced, she just changed affiliations, so it might be related to that.
\item Sara Stymne, Uppsala University, Uppsala, Sweden %SL: currently on parental leave
\item Jörg Tiedemann, University of Helsinki, Finland (confirmed)
\item Yannick Versley, University of Heidelberg, Heidelberg, Germany % no longer at Heidelberg, indep consultant, not clear if he still wants to review for DiscoMT
\item Martin Volk, University of Zurich, Zurich, Switzerland (confirmed)
\item Bonnie Webber, University of Edinburgh, UK
\item Kellie Webster, Google, New York, US (confirmed)
\item Min Zhang, Soochow University, Suzhou, China (confirmed)
\item Sandrine Zufferey, University of Bern, Switzerland (confirmed)
\end{itemize}

% do we want to include:
% Guillaume Lample (PhD student at UPMC, also at Facebook, etc,) -> SL: not sure he does discourse, APB: not really, but working on NMT, so he could review papers on using larger contexts in NMT ("document-level")
% Arianna Bisazza (or Luisa Bentivogli?) -> SL: they don't seem to do discourse really APB: OK (but they did nice evaluation work)

\section{Invited Speakers} % and sources for funding}

Invited speakers: Kellie Webster (Google), and Andy Way (Dublin).  Include their biographies. % and funding?


\section{Diversity}

% Account of efforts for diversity of speakers, program committee members and participants
% Balance invited speakers
% Balance committee: gender, geography, community is very Europe-oriented, but active efforts to recruit people from outside Europe, compared to previous years.
% Gender balance in PCs: female/male: 7/18 in 2013 (total 25), 11/11 in 2015 (total 22), 9/14 (total 23).  Quite good!  And this year: 16/14 approximate!

From the beginning of the DiscoMT we strive to maintain an appropriate gender balance, geographical representativeness, but also topical inclusion among the discourse studies, NLP and MT communities.  Although the first initiative for DiscoMT emerged in Europe, the program committee has always included representatives from the other continents: America (limited to Canada and the USA), the Far-East (China, Singapore), and the Middle-East (Qatar).  We are also attentive to the balance between universities and corporate research centers.  This year, the two invited speakers contribute to the diversity (gender, geography, institution), and the program committee will, for the first time, include more women than men.  Our statistics run indeed as follows for the female/male ratio of PC members: 7/18 in 2013, 11/11 in 2015, 9/14 in 2017, and about 16/14 in 2019.



\section{Other Information}

\begin{itemize}
\item \textbf{Proposed workshop length:} One day.  Ideally, this would coincide with
the second day of the Conference on Machine Translation (WMT), so that the two workshops could
share a poster session, as has been the case in the past editions of DiscoMT.

\item \textbf{Estimated number of attendees:} 25-30 people

\item \textbf{Special requirements or technical needs:} None

\item \textbf{Preferred venue:} our preferred venue is ACL 2019 in Florence, Italy, for several reasons.  The first one is that the center of gravity of the DiscoMT community is in Europe, and we foresee that a venue in Europe will minimize travel costs and may maximize participation.  The second one is that DiscoMT has close ties with the Conference in Machine Translation (WMT), including the tradition of organizing a joint poster session, and we aim to be co-located with WMT (which aims for ACL in Florence as well).  The second preferred venue is NAACL in Minneapolis, MI, USA.

\end{itemize}




% -----------------------------------------------------------
\section{Previous Editions of DiscoMT: 2013, 2015 and 2017}

% \textcolor{magenta}{Include stats about how many participants we had before. Include DiscoMT2017.} 
% APB: shorten and try to do a table with the numbers 

\textbf{Previous editions.}  The DiscoMT series has been organized every two years, starting in 2013.  The growing interest in discourse and MT led to the first Workshop on Discourse in Machine Translation (DiscoMT) in 2013, held as a satellite workshop of ACL in Sofia, Bulgaria, at a time when statistical phrase-based or hierarchical MT was the state of the art.\footnote{Bonnie Webber  and  Andrei Popescu-Belis  and  Katja Markert  and  J\"{o}rg Tiedemann. (2013). \textit{Proceedings of the Workshop on Discourse in Machine Translation (DiscoMT)}. Sofia, Bulgaria. Association for Computational Linguistics. \url{http://aclweb.org/anthology/W13-3300}.} 
The second DiscoMT was held in 2015 in Lisbon, Portugal, in conjunction with EMNLP.\footnote{Webber, B., Carpuat, M., Popescu-Belis, A., \& Hardmeier, C. (2015). \textit{Proceedings of the Second Workshop on Discourse in Machine Translation (DiscoMT 2015)}. Lisbon, Portugal. Association for Computational Linguistics.  \url{http://aclweb.org/anthology/W15-2500}.}  This workshop featured a shared task on pronoun translation, which had a significant impact on the community as the resources and methodology were developed and reused up until day (see below).
The third DiscoMT workshop, in 2017, was again a satellite of EMNLP and took place in Copenhagen\footnote{Webber, B., Popescu-Belis, A., \& Tiedemann, J. (2017). \textit{Proceedings of the Third Workshop on Discourse in Machine Translation (DiscoMT 2017)}. Copenhagen, Denmark. Association for Computational Linguistics.  \url{http://aclweb.org/anthology/W17-4800}.}

At all the three previous DiscoMT workshops, oral presenters were also encouraged to make posters
describing their work, and all resulting posters were presented in a joint
poster session with the Conference on Machine Translation (WMT), thus offering DiscoMT an increased visibility.

\textbf{Statistics.}  The DiscoMT have been rather small, albeit inclusive workshops, devoting significant time to discussions, and welcoming novel ideas at an early stage.  The number of papers has evolved as follows:
\begin{itemize}
    \item \textbf{2013:} 12 submissions, 6 accepted as oral presentations and 2 as posters;
    \item \textbf{2015:} 15 submissions, 5 accepted as long papers, 3 as short papers, and 4 as posters; in addition, there were 8 system posters and 1 overview of the shared task;
    \item \textbf{2017:} 14 submissions, 4 accepted as long talks, 3 as short talks, and 2 as posters; in addition, there were 4 system posters and 1 overview of the shared task.
\end{itemize}


\textbf{Shared tasks.}  Shared tasks on discourse in MT have been organized at DiscoMT 2015 and 2017, focussing on pronoun translation.  Moreover, a similar shared task was organized in-between, at WMT 2016, and data for evaluation has led to testing on an ``extra test suite'' at WMT 2018.  Given the mechanism put into place in 2018, we do not foresee a separate shared task at DiscoMT 2019, but encourage discourse-level testing at WMT 2019 (hence the importance of a joint poster session with DiscoMT).  The impact of all these tasks on the field has been quite considerable, and one of the contributions of DiscoMT to the field.

\begin{enumerate}
    \item The DiscoMT 2015 shared task\footnote{Christian Hardmeier, Preslav Nakov, Sara Stymne, Jörg Tiedemann, Yannick Versley, and Mauro Cettolo. (2015). Pronoun-Focused MT and Cross-Lingual Pronoun Prediction: Findings of the 2015 DiscoMT Shared Task on Pronoun Translation.  \textit{Proceedings of the Second Workshop on Discourse in Machine Translation (DiscoMT)}, pages 1–16, Lisbon, Portugal, 17 September 2015.  Association for Computational Linguistics.  \url{http://aclweb.org/anthology/W15-2501}} included two subtasks, relevant to both the MT and the discourse communities: pronoun-focused translation, and cross-lingual pronoun prediction.  Six groups participated in the pronoun-focused translation task and eight groups in the cross-lingual pronoun prediction task.
    
    \item The WMT 2016 shared task on cross-lingual pronoun prediction\footnote{Guillou, Liane; Hardmeier, Christian; Nakov, Preslav; Stymne, Sara; Tiedemann, Jörg; Versley, Yannick; Cettolo, Mauro; Webber, Bonnie; Popescu-Belis, Andrei. (2016).  Findings of the 2016 WMT shared task on cross-lingual pronoun prediction, \textit{Proceedings of the First Conference on Machine Translation (WMT16)}, pages 522-539.  Berlin, Germany. Association for Computational Linguistics. \url{http://aclweb.org/anthology/W16-2345}} was a  classification  task  in which  participants  were  asked  to  provide predictions  on  what  pronoun  class  label should replace a placeholder value in the target-language  text,  provided  in  lemmatised and PoS-tagged form.  Eleven  teams  participated  in the  shared  task.
    
    \item The DiscoMT 2017 shared task\footnote{Lo{\'a}iciga, Sharid; Stymne, Sara; Nakov, Preslav; Hardmeier, Christian; Tiedemann, J{\"o}rg; Cettolo, Mauro; Versley, Yannick. (2017). Findings of the 2017 DiscoMT Shared Task on Cross-lingual Pronoun Prediction, \textit{Proceedings of the Third Workshop on Discourse in Machine Translation}, pages 1--16. Copenhagen, Denmark. Association for Computational Linguistics.  \url{http://aclweb.org/anthology/W17-4801}} aimed again at pronoun prediction (given the source text and a lemmatized (or ``impoverished'') version of the target text stripped of its pronouns) with four language pairs.   Five teams participated in the 2017 shared task.
    
    \item Finally, the outputs of the news translation systems at WMT 2018 have been evaluated on a pronoun test set.\footnote{Guillou, Liane; Hardmeier, Christian; Lapshinova-Koltunski, Ekaterina; Lo\'{a}iciga, Sharid.  (2018). A Pronoun Test Suite Evaluation of the English--German MT Systems at WMT 2018.  \textit{Proceedings of the Third Conference on Machine Translation}, pages 576--583. Belgium, Brussels.  Association for Computational Linguistics. \url{http://www.aclweb.org/anthology/W18-64062}}
\end{enumerate}


\end{document}